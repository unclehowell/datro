%% Generated by Sphinx.
\def\sphinxdocclass{report}
\documentclass[letterpaper,10pt,openany,oneside,english]{sphinxmanual}
\ifdefined\pdfpxdimen
   \let\sphinxpxdimen\pdfpxdimen\else\newdimen\sphinxpxdimen
\fi \sphinxpxdimen=.75bp\relax

\PassOptionsToPackage{warn}{textcomp}
\usepackage[utf8]{inputenc}
\ifdefined\DeclareUnicodeCharacter
% support both utf8 and utf8x syntaxes
  \ifdefined\DeclareUnicodeCharacterAsOptional
    \def\sphinxDUC#1{\DeclareUnicodeCharacter{"#1}}
  \else
    \let\sphinxDUC\DeclareUnicodeCharacter
  \fi
  \sphinxDUC{00A0}{\nobreakspace}
  \sphinxDUC{2500}{\sphinxunichar{2500}}
  \sphinxDUC{2502}{\sphinxunichar{2502}}
  \sphinxDUC{2514}{\sphinxunichar{2514}}
  \sphinxDUC{251C}{\sphinxunichar{251C}}
  \sphinxDUC{2572}{\textbackslash}
\fi
\usepackage{cmap}
\usepackage[T1]{fontenc}
\usepackage{amsmath,amssymb,amstext}
\usepackage[english]{babel}



\usepackage{times}
\expandafter\ifx\csname T@LGR\endcsname\relax
\else
% LGR was declared as font encoding
  \substitutefont{LGR}{\rmdefault}{cmr}
  \substitutefont{LGR}{\sfdefault}{cmss}
  \substitutefont{LGR}{\ttdefault}{cmtt}
\fi
\expandafter\ifx\csname T@X2\endcsname\relax
  \expandafter\ifx\csname T@T2A\endcsname\relax
  \else
  % T2A was declared as font encoding
    \substitutefont{T2A}{\rmdefault}{cmr}
    \substitutefont{T2A}{\sfdefault}{cmss}
    \substitutefont{T2A}{\ttdefault}{cmtt}
  \fi
\else
% X2 was declared as font encoding
  \substitutefont{X2}{\rmdefault}{cmr}
  \substitutefont{X2}{\sfdefault}{cmss}
  \substitutefont{X2}{\ttdefault}{cmtt}
\fi


\usepackage[Bjarne]{fncychap}
\usepackage{sphinx}

\fvset{fontsize=\small}
\usepackage{geometry}


% Include hyperref last.
\usepackage{hyperref}
% Fix anchor placement for figures with captions.
\usepackage{hypcap}% it must be loaded after hyperref.
% Set up styles of URL: it should be placed after hyperref.
\urlstyle{same}


\usepackage{sphinxmessages}
\setcounter{tocdepth}{1}



\title{HBnB - Operation Federate}
\date{Aug 29, 2021}
\release{0.0.0}
\author{Author(s): DATRO Consortium}
\newcommand{\sphinxlogo}{\vbox{}}
\renewcommand{\releasename}{User Manual | Version}
\makeindex
\begin{document}

\pagestyle{empty}
\sphinxmaketitle
\pagestyle{plain}
\sphinxtableofcontents
\pagestyle{normal}
\phantomsection\label{\detokenize{index::doc}}


\sphinxAtStartPar
Index:


\chapter{Release Notes and Notices}
\label{\detokenize{releasenotes:release-notes-and-notices}}\label{\detokenize{releasenotes::doc}}
\sphinxAtStartPar
This section provides information about what is new or changed, including urgent issues, documentation updates, maintenance, and new releases.
\sphinxhyphen{} ‘Updates’ are the term used to describe significant changes to our public source code and/or records.


\section{This Release (Version 0.0.0)}
\label{\detokenize{releasenotes:this-release-version-0-0-0}}\begin{itemize}
\item {} 
\sphinxAtStartPar
\sphinxstylestrong{2021\sphinxhyphen{}Jun\sphinxhyphen{}24} \sphinxhyphen{} \sphinxtitleref{video published for the idea to create HotspotBnB “Plus”}

\end{itemize}


\section{Older Versions}
\label{\detokenize{releasenotes:older-versions}}
\sphinxAtStartPar
In the table below the last entry displays a link to an archived copy of the last report.
To keep the filename from overflowing in the table below the name displayed may differ from the file name.
The date the file was archived will differ from the date of the document label, which is its creation date.
If you’re viewing this document on a subdomain of \sphinxtitleref{.datro.world} you may need to right\sphinxhyphen{}click and select ‘open link in new tab\textasciigrave{}.
In the interim of a bug fix, you can avoid right\sphinxhyphen{}clicking all together, by viewing our document library at its original location \sphinxhref{https://datro.xyz/static/library}{datro.xyz/static/library}


\begin{savenotes}\sphinxattablestart
\centering
\sphinxcapstartof{table}
\sphinxthecaptionisattop
\sphinxcaption{Older Versions of this Document}\label{\detokenize{releasenotes:id1}}
\sphinxaftertopcaption
\begin{tabular}[t]{|\X{20}{100}|\X{20}{100}|\X{20}{100}|\X{40}{100}|}
\hline
\sphinxstyletheadfamily 
\sphinxAtStartPar
\sphinxstylestrong{Archive Date}
&\sphinxstyletheadfamily 
\sphinxAtStartPar
\sphinxstylestrong{Version}
&\sphinxstyletheadfamily 
\sphinxAtStartPar
\sphinxstylestrong{Description}
&\sphinxstyletheadfamily 
\sphinxAtStartPar
\sphinxstylestrong{Download Link}
\\
\hline
\sphinxAtStartPar
n/a
&
\sphinxAtStartPar
n/a
&
\sphinxAtStartPar
n/a
&
\sphinxAtStartPar
n/a
\\
\hline
\end{tabular}
\par
\sphinxattableend\end{savenotes}


\section{Known and Corrected Issues}
\label{\detokenize{releasenotes:known-and-corrected-issues}}
\sphinxAtStartPar
Below is a table of pending issues which have been reported to our team.
When these issues are remedied, or any significant changed made, a new release will be published.


\begin{savenotes}\sphinxattablestart
\centering
\sphinxcapstartof{table}
\sphinxthecaptionisattop
\sphinxcaption{Known Issues}\label{\detokenize{releasenotes:id2}}
\sphinxaftertopcaption
\begin{tabular}[t]{|\X{20}{100}|\X{15}{100}|\X{25}{100}|\X{40}{100}|}
\hline
\sphinxstyletheadfamily 
\sphinxAtStartPar
\sphinxstylestrong{Date}
&\sphinxstyletheadfamily 
\sphinxAtStartPar
\sphinxstylestrong{Version}
&\sphinxstyletheadfamily 
\sphinxAtStartPar
\sphinxstylestrong{Subject}
&\sphinxstyletheadfamily 
\sphinxAtStartPar
\sphinxstylestrong{Description}
\\
\hline
\sphinxAtStartPar
n/a
&
\sphinxAtStartPar
n/a
&
\sphinxAtStartPar
n/a
&
\sphinxAtStartPar
n/a
\\
\hline
\end{tabular}
\par
\sphinxattableend\end{savenotes}


\chapter{Executive Summary}
\label{\detokenize{executivesummary:executive-summary}}\label{\detokenize{executivesummary::doc}}
\sphinxAtStartPar
DATRO is a for\sphinxhyphen{}profit consortium of technology solutions, designed to help end\sphinxhyphen{}users transition to the next generation of internet (web3.0).
Our primary software solution (Hotspotβnβ) is an operating system for mini IoT home servers connects wirelessly to end\sphinxhyphen{}users existing Wi\sphinxhyphen{}Fi access point.
Hotspotβnβ will eventually become universal enough to be loaded directly onto wireless access points as a free software upgrade.
Unlike Android, exploring and installing applications with Hotspotβnβ, isn’t limited to a single session and screen.
Therefore any number of desktops, laptops, tablets or mobile devices connecting to the access point, can enjoy the apps via their web\sphinxhyphen{}browser.

\sphinxAtStartPar
Until now the webapps running on Hotspotβnβ have primarily orbited around the theme of smart home control and home entertainment (including gaming).
Benefits include multi\sphinxhyphen{}room/multi\sphinxhyphen{}user access, offline functionality and complete elimination of big tech, attempting to regulate your in\sphinxhyphen{}home activities.
Business, Communications and productivity webapps also exist, but until now teams working from home have prefered big tech services e.g. salesforce, xero etc
However a new market is emerging (with 2Million users in 2021) which offers the independance of hosting apps yourself,
with all the flexibilities of accessing these apps from anywhere (subscription\sphinxhyphen{}free) and collaborating on them as teams, from various locations.
This new market segmenent is entitled the Fediverse (Federated Universe) and includes decentralized and federated social media, file sharing and more.

\sphinxAtStartPar
Localhost, Decentralized, Federated, Progressive WebApps is the future. Let’s break down what these mean:

\sphinxAtStartPar
1. \sphinxstylestrong{Progressive WebApps:} The overal experience of Progressive WebApps feels like your wi\sphinxhyphen{}fi signal (distance/range) is much greater, because going out of range or offline doesn’t result in any significant loss of functionality or service.
So in the case of business, you can work from home, all alone, using your low\sphinxhyphen{}budget consumer device, low internet speed and poor wi\sphinxhyphen{}fi signal,
And still feel like you’re in the office, with your team, on an enterprise grade work station and high\sphinxhyphen{}speed network.

\sphinxAtStartPar
2. \sphinxstylestrong{Localhost:} The applications are hosted on your own mini webserver (a mere \$30 \sphinxhyphen{} \$200 USD), which connects to the back of your wireless access point.
So internet or not, you’re always up and running at your home office. And peace of mind security is great because no data leaves your home.

\sphinxAtStartPar
3. \sphinxstylestrong{Decentralized:} Your team no longer rely on a centralized service or server. Your organisation has become de\sphinxhyphen{}centralized while working from home.
This approach to decentralizing aligns perfectly with the strong emergence of cryptocurrency and blockchain protocols.

\sphinxAtStartPar
4. \sphinxstylestrong{Federating:} Imagine a business owner, working from home, using a federated CRM/ERP Application.
The business has grown and others join the company, also working from home.
Each home\sphinxhyphen{}based worker can also install a copy of the CRM Application and 100\% own and control that data.
But instead of operating independantly of each other, a team can synchronise, peer\sphinxhyphen{}to\sphinxhyphen{}peer.
This is called Federating e.g. a “Federated WebApp”.

\sphinxAtStartPar
Adoption of these above points (with the exclusion of Federating) has been largely successful and popular in smart homes running homeservers.
And in the case of businesses running their own servers at their offices.
But with the recent rise of working from home, it’s long time businesses decentralize their offices servers across their employees homeservers.
Equally business using centralized online services are expected to move to decentralized homeservers in light of the emegence of federated services on the backend and progressive webapps on the front\sphinxhyphen{}end.
Hotspotβnβ has already been in this space for a while with smarthome apps. With some small alterations there’s no reason the software cannot support business apps and users.


\section{Challenge}
\label{\detokenize{executivesummary:challenge}}
\sphinxAtStartPar
1. With centralized online services (particularly in business) there’s a huge amount of convenience and support. End\sphinxhyphen{}Users don’t have to think about backups, server uptime, resetting passwords etc.
When you take on all of this responsibility for yourself, outsourcing with a subscription fee seems feasible.


\section{Solution \& Outcomes}
\label{\detokenize{executivesummary:solution-outcomes}}\begin{enumerate}
\sphinxsetlistlabels{\arabic}{enumi}{enumii}{}{.}%
\item {} 
\sphinxAtStartPar
Centralized data and power has is in itself demonstrated huge flaws in recent years:

\end{enumerate}
\begin{itemize}
\item {} 
\sphinxAtStartPar
the service provider can deny a company access to all its data at anytime for any number of reasons, payment, terms etc

\item {} 
\sphinxAtStartPar
centralized service providers storing multiple companies data are targets for hacks and data leaks

\item {} 
\sphinxAtStartPar
even with the recent introduction of Progressive Web Apps, any issues with local internet access means temporary loss of access

\item {} 
\sphinxAtStartPar
big data farming e.g. you might spot patterns and trade secrets for key business decisions in all your organisations data,
but so too do the centralized service providers, who sell that info, meaning market opportunities are exploited ahead of your company

\end{itemize}

\sphinxAtStartPar
In many ways centralized services are a relic of a time gone by.
They’re maturity while also facing extinction, now makes them wolves in sheeps clothing.
Centralied service providers signified the potential of the first 2 decades of the 21st Century, but that’s now over.
Again the rich got richer and the little people got further disempowered. Web 2.0 failed. Web 3.0 will not.

\sphinxAtStartPar
Small businesses are wising up and so too are their clientel.
It’s now become somewhat of a deterrant, to anyone who is anyone, to even engage with anyone still using gmail instead of proton or fiat currencies instead of cryptocurrencies

\sphinxAtStartPar
Hotspotβnβ is providing the same benefits of a centralized service but for decentralized, federated, localhost apps.
The software is able to perform Over The Air (OTA) updates and upgrades.
And the software and applications are supported by hundreds of thousands of developers.

\sphinxAtStartPar
While it’s important to still cater to clients who still use centralized methods, using decentralized sovereign solutions is the new “mobile first”.


\section{Market Potential}
\label{\detokenize{executivesummary:market-potential}}
\sphinxAtStartPar
DATRO’s approach is to sell 50\% co\sphinxhyphen{}ownership (co\sphinxhyphen{}lessor rights) for a one time fixed sum of \$500 USD per 460 meter honeycomb.
For this price the Scottish Bay alone will generate in the region of circa \$1.4M USD of digital real estate inventory for the consortium to sell.
The area is defined by the 70km width of the coastline and the depth is inland as far as the west/southern boundaries of the two provinces which make up the Scottish Bay.
The services and nodes inside each dome effectively become lessee’s, using cryptocurrency smart contracts to lease use and enjoyment of the Bloculus protocol.
The proceeds of which will pool together and divide proportionately between the protocols lessors/ beneficiaries.
Henceforth the Scottish Bay will become the first of many estates, of this new protocol.
Furthermore, the web3.0 services and nodes on this new fixed communications network, are expected to be the first of many types of digital lessee’s which will pay to use the Bloculus protocol.


\section{Recommendation}
\label{\detokenize{executivesummary:recommendation}}
\sphinxAtStartPar
It’s recommended the consortium expand on this business case and produce a dedicated whitepaper on the technology.
It would be wise to purchase the Scottish Bay’s entire H3 references as TLD’s, ahead of competing party.
Then select future regions of the world to ringfence H3/TLD’s for use with this protocol.
The retail rate of co\sphinxhyphen{}ownership of the preceeding estate, should help towards this growing capital requirement, failing this investors maybe interested in financing this initiative.

\sphinxAtStartPar
Moving forward the consortium will overlay the desired H3 grid onto a map of the Scottish Bay.
A salesforce will sell the inventory to interested parties.
The lessor/lessee agreement will be as a cryptocurrency smart contract.
\begin{description}
\item[{A typical business case would see this estate divided into 2,800 honeycombs.}] \leavevmode\begin{enumerate}
\sphinxsetlistlabels{\alph}{enumi}{enumii}{}{)}%
\item {} 
\sphinxAtStartPar
In this example a capital investor purchases co\sphinxhyphen{}ownership of a quarter of this estate (700 H3 hexagons) for circa \$350,000 USD (\$500 per honeycomb).

\item {} 
\sphinxAtStartPar
The first lessee is the aformentioned mesh network, which takes say 3 years to construct and begins earning in the region of \$6M USD per annum for use/enjoyment of the protocol.

\item {} 
\sphinxAtStartPar
The benefactor which co\sphinxhyphen{}owns a quarter of the estate would receive \$750,000 USD per annum in royalties.

\item {} 
\sphinxAtStartPar
At this juncture the currency invested would be USD but the currency generated by the network and paid to beneficiaries would be the cryptocurrency DOT (on the Polkadot blockchain).

\end{enumerate}

\end{description}

\sphinxAtStartPar
The interface for configuring the final phase of deployment of the domes is HotspotBnB. A simple localhost webapp, developed by the DATRO Consortium.
HotspotBnB features a built in appstore which supports ‘one\sphinxhyphen{}click’ install of a variety of software (including DApps) for uniformity and scaliability.
HotspotBnB is ultimately just a webserver designed to run on a low energy/ low cost single board computers e.g. Raspberry Pi.
The operating system autonomously self\sphinxhyphen{}builds and configures. And can do so without an active internet connection (using another DATRO solution called Cacher)

\sphinxAtStartPar
DATRO is soon to release a self\sphinxhyphen{}service website for making customisations to this autonomous self\sphinxhyphen{}buiding OS prior downloading a copy (websites are also all accessible offline via Cacher)
HotspotBnB can be used as a residential wireless IoT Home Server if the end\sphinxhyphen{}user enters their wireless router SSID and password before generating their copy of the OS.
HotspotBnB can also be used to manage a Geodesic Equipment Room if the physical location (in H3/ resolution 8 format) is pre\sphinxhyphen{}selected in order to include the H3\sphinxhyphen{}TLD inside the OS.
Now when HotspotBnB is booted up (providing an active internet exists or Cacher is used to simulate internet) it can identify itself and pair to other equipment rooms in its proximity.


\section{Justification}
\label{\detokenize{executivesummary:justification}}
\sphinxAtStartPar
This protocol is justified from both a technical and business standpoint.
The alternatives aren’t half as effective and have limitations which this protocol overcomes.
Furthermore this is a new generation of communications network and so a new protocol has had to be developed specifically because the existing technologies didn’t suffice.
The Bloculus protocol is designed for a reality of automation, decentralization, anonymity, cryptographic security, currency and tokenized voting/ liquid democracy.


\section{Annexures}
\label{\detokenize{executivesummary:annexures}}
\sphinxAtStartPar
A suppliment or appendix to a written document. An annexure is an addition to something, often to a document.
When used generally to simply mean something added, annexure is interchangeable with annex. More commonly used in Britain and India, where it often specifically refers to an addition to an official document.


\chapter{Finance}
\label{\detokenize{finance:finance}}\label{\detokenize{finance::doc}}
\sphinxAtStartPar
Excepteur sint occaecat cupidatat non proident, sunt in culpa qui officia deserunt mollit anim id est laborum.
Lorem ipsum dolor sit amet, consectetur adipiscing elit, sed do eiusmod tempor incididunt ut labore et dolore magna aliqua. Ut enim ad minim veniam, quis nostrud exercitation ullamco laboris nisi ut aliquip ex ea commodo consequat


\chapter{Project Definition}
\label{\detokenize{projectdefinition:project-definition}}\label{\detokenize{projectdefinition::doc}}

\section{Background information}
\label{\detokenize{projectdefinition:background-information}}

\section{Business Objective}
\label{\detokenize{projectdefinition:business-objective}}

\section{Benefits and Limitations}
\label{\detokenize{projectdefinition:benefits-and-limitations}}

\section{Option Identification \& Selection}
\label{\detokenize{projectdefinition:option-identification-selection}}

\section{Scope, Impact, and interdependencies}
\label{\detokenize{projectdefinition:scope-impact-and-interdependencies}}

\section{Outline Plan}
\label{\detokenize{projectdefinition:outline-plan}}

\section{Market Assessment}
\label{\detokenize{projectdefinition:market-assessment}}

\section{Risk Assessment}
\label{\detokenize{projectdefinition:risk-assessment}}

\section{Project Approach}
\label{\detokenize{projectdefinition:project-approach}}

\section{Purchasing Strategy}
\label{\detokenize{projectdefinition:purchasing-strategy}}

\chapter{Installation Guide}
\label{\detokenize{installation:installation-guide}}\label{\detokenize{installation::doc}}

\chapter{General Usage}
\label{\detokenize{end-user:general-usage}}\label{\detokenize{end-user::doc}}

\section{What is Hotspotβnβ?}
\label{\detokenize{end-user:what-is-hotspotn}}
\sphinxAtStartPar
The Hotspotβnβ Operating System (Hotspotβnβ) is a Free \& Open\sphinxhyphen{}Source Linux\sphinxhyphen{}Based Software, designed to make any Single Board Computer a Plug \& Play \sphinxstylestrong{Smart Smart Home Hotspot} with the unique capability of making internet Freer and/or completely free for the household in which it operates. The solution also features some great open source and free apps for the end\sphinxhyphen{}users e.g. IPTV/ Media Center, IPCCTV DVR, IoT Smart Home Control, Vehicle Tracking and Energy Monitoring \sphinxhyphen{} All autonomously installed and configured during initial installation, which is also autonomous. Hotspotβnβ simply copies onto a Micro SD card, inserts into any Single Board Computer connected to the internet and within minutes can be used, enjoyed and benefitted from.


\section{Disclaimer}
\label{\detokenize{end-user:disclaimer}}
\sphinxAtStartPar
Keep in mind that although I am a professional engineer with extensive background experience and education, this is the first product of this magnitude I have attempted to develop, the work is ongoing and as of 2018 I am still in the early phases and some time away from a final solution. There is much I have yet to learnt about best practices for Documenting, Open Source Code \& version Control and Systems Integration of the various technologies included in this new Operating System. There are industry standard methodologies this new Operating System is yet to adhear to \sphinxhyphen{} I remain humble and open to suggestions at all levels.

\sphinxAtStartPar
Everything you find here at this stage is without waranty and I accept not responsible for any inconveniences or issues that might occur as a result of use of this new Operating System: Hotspotβnβ. As time goes on I can only assure you that less Single Board Computers and Memory Sticks become damaged and/or ‘bricked’ by Hotspotβnβ. Fortunately both are low cost hardware and Hotspotβnβ is free, so it’s feasible fun and promising technology right now, to say the least.


\section{Languages}
\label{\detokenize{end-user:languages}}
\sphinxAtStartPar
The primary language of Hotspotβnβ will be English. Secondary languages will be introduced using translators which will use the English literature as the primary source of information.


\chapter{Device Upkeep}
\label{\detokenize{productmaint:device-upkeep}}\label{\detokenize{productmaint::doc}}

\section{hyperlink}
\label{\detokenize{productmaint:hyperlink}}
\sphinxAtStartPar
Hyperlink \sphinxhref{https://ArchLinuxarm.org/platforms/armv6/raspberry-pi}{here},


\section{command lines}
\label{\detokenize{productmaint:command-lines}}
\sphinxAtStartPar
like this: \sphinxcode{\sphinxupquote{192.168.0.x}}, \sphinxcode{\sphinxupquote{10.0.0.14x}} or

\sphinxAtStartPar
Enough of networking for now. We’ll set a proper network configuration later in this guide, but first some \sphinxstyleemphasis{musthaves}.


\subsection{text block}
\label{\detokenize{productmaint:text-block}}
\begin{sphinxVerbatim}[commandchars=\\\{\}]
\PYG{n}{passwd}  \PYG{c+c1}{\PYGZsh{} change root password to something important}
\PYG{n}{rm} \PYG{o}{\PYGZhy{}}\PYG{n}{rf} \PYG{o}{/}\PYG{n}{etc}\PYG{o}{/}\PYG{n}{localtime}  \PYG{c+c1}{\PYGZsh{} dont care about this}
\PYG{n}{ln} \PYG{o}{\PYGZhy{}}\PYG{n}{s} \PYG{o}{/}\PYG{n}{usr}\PYG{o}{/}\PYG{n}{share}\PYG{o}{/}\PYG{n}{zoneinfo}\PYG{o}{/}\PYG{n}{Europe}\PYG{o}{/}\PYG{n}{Prague} \PYG{o}{/}\PYG{n}{etc}\PYG{o}{/}\PYG{n}{localtime}  \PYG{c+c1}{\PYGZsh{} set appropriate timezone}
\PYG{n}{echo} \PYG{l+s+s2}{\PYGZdq{}}\PYG{l+s+s2}{my\PYGZus{}raspberry}\PYG{l+s+s2}{\PYGZdq{}} \PYG{o}{\PYGZgt{}}  \PYG{o}{/}\PYG{n}{etc}\PYG{o}{/}\PYG{n}{hostname}  \PYG{c+c1}{\PYGZsh{} set name of your RPi}

\PYG{n}{useradd} \PYG{o}{\PYGZhy{}}\PYG{n}{m} \PYG{o}{\PYGZhy{}}\PYG{n}{aG} \PYG{n}{wheel} \PYG{o}{\PYGZhy{}}\PYG{n}{s} \PYG{o}{/}\PYG{n}{usr}\PYG{o}{/}\PYG{n+nb}{bin}\PYG{o}{/}\PYG{n}{bash} \PYG{n}{common\PYGZus{}user} \PYG{c+c1}{\PYGZsh{}}
\PYG{n}{groupadd} \PYG{n}{webdata}  \PYG{c+c1}{\PYGZsh{} for sharing}
\PYG{n}{useradd} \PYG{o}{\PYGZhy{}}\PYG{n}{M} \PYG{o}{\PYGZhy{}}\PYG{n}{aG} \PYG{n}{webdata} \PYG{o}{\PYGZhy{}}\PYG{n}{s} \PYG{o}{/}\PYG{n}{usr}\PYG{o}{/}\PYG{n+nb}{bin}\PYG{o}{/}\PYG{n}{false} \PYG{n}{nginx}
\PYG{n}{usermod} \PYG{o}{\PYGZhy{}}\PYG{n}{aG} \PYG{n}{webdata} \PYG{n}{common\PYGZus{}user}

\PYG{n}{visudo}  \PYG{c+c1}{\PYGZsh{} uncomment this line:  \PYGZpc{}wheel ALL=(ALL) ALL}

\PYG{n}{pacman} \PYG{o}{\PYGZhy{}}\PYG{n}{Syu}
\end{sphinxVerbatim}

\sphinxAtStartPar
\sphinxstylestrong{bold text}
\begin{itemize}
\item {} 
\sphinxAtStartPar
bullet

\item {} 
\sphinxAtStartPar
point

\end{itemize}


\chapter{Preferences}
\label{\detokenize{afterinstallconfig:preferences}}\label{\detokenize{afterinstallconfig::doc}}

\section{hyperlink}
\label{\detokenize{afterinstallconfig:hyperlink}}
\sphinxAtStartPar
Hyperlink \sphinxhref{https://ArchLinuxarm.org/platforms/armv6/raspberry-pi}{here},


\section{command lines}
\label{\detokenize{afterinstallconfig:command-lines}}
\sphinxAtStartPar
like this: \sphinxcode{\sphinxupquote{192.168.0.x}}, \sphinxcode{\sphinxupquote{10.0.0.14x}} or

\sphinxAtStartPar
Enough of networking for now. We’ll set a proper network configuration later in this guide, but first some \sphinxstyleemphasis{musthaves}.


\subsection{text block}
\label{\detokenize{afterinstallconfig:text-block}}
\begin{sphinxVerbatim}[commandchars=\\\{\}]
\PYG{n}{passwd}  \PYG{c+c1}{\PYGZsh{} change root password to something important}
\PYG{n}{rm} \PYG{o}{\PYGZhy{}}\PYG{n}{rf} \PYG{o}{/}\PYG{n}{etc}\PYG{o}{/}\PYG{n}{localtime}  \PYG{c+c1}{\PYGZsh{} dont care about this}
\PYG{n}{ln} \PYG{o}{\PYGZhy{}}\PYG{n}{s} \PYG{o}{/}\PYG{n}{usr}\PYG{o}{/}\PYG{n}{share}\PYG{o}{/}\PYG{n}{zoneinfo}\PYG{o}{/}\PYG{n}{Europe}\PYG{o}{/}\PYG{n}{Prague} \PYG{o}{/}\PYG{n}{etc}\PYG{o}{/}\PYG{n}{localtime}  \PYG{c+c1}{\PYGZsh{} set appropriate timezone}
\PYG{n}{echo} \PYG{l+s+s2}{\PYGZdq{}}\PYG{l+s+s2}{my\PYGZus{}raspberry}\PYG{l+s+s2}{\PYGZdq{}} \PYG{o}{\PYGZgt{}}  \PYG{o}{/}\PYG{n}{etc}\PYG{o}{/}\PYG{n}{hostname}  \PYG{c+c1}{\PYGZsh{} set name of your RPi}

\PYG{n}{useradd} \PYG{o}{\PYGZhy{}}\PYG{n}{m} \PYG{o}{\PYGZhy{}}\PYG{n}{aG} \PYG{n}{wheel} \PYG{o}{\PYGZhy{}}\PYG{n}{s} \PYG{o}{/}\PYG{n}{usr}\PYG{o}{/}\PYG{n+nb}{bin}\PYG{o}{/}\PYG{n}{bash} \PYG{n}{common\PYGZus{}user} \PYG{c+c1}{\PYGZsh{}}
\PYG{n}{groupadd} \PYG{n}{webdata}  \PYG{c+c1}{\PYGZsh{} for sharing}
\PYG{n}{useradd} \PYG{o}{\PYGZhy{}}\PYG{n}{M} \PYG{o}{\PYGZhy{}}\PYG{n}{aG} \PYG{n}{webdata} \PYG{o}{\PYGZhy{}}\PYG{n}{s} \PYG{o}{/}\PYG{n}{usr}\PYG{o}{/}\PYG{n+nb}{bin}\PYG{o}{/}\PYG{n}{false} \PYG{n}{nginx}
\PYG{n}{usermod} \PYG{o}{\PYGZhy{}}\PYG{n}{aG} \PYG{n}{webdata} \PYG{n}{common\PYGZus{}user}

\PYG{n}{visudo}  \PYG{c+c1}{\PYGZsh{} uncomment this line:  \PYGZpc{}wheel ALL=(ALL) ALL}

\PYG{n}{pacman} \PYG{o}{\PYGZhy{}}\PYG{n}{Syu}
\end{sphinxVerbatim}

\sphinxAtStartPar
\sphinxstylestrong{bold text}
\begin{itemize}
\item {} 
\sphinxAtStartPar
bullet

\item {} 
\sphinxAtStartPar
point

\end{itemize}


\chapter{Integration}
\label{\detokenize{integration:integration}}\label{\detokenize{integration::doc}}

\section{hyperlink}
\label{\detokenize{integration:hyperlink}}
\sphinxAtStartPar
Hyperlink \sphinxhref{https://ArchLinuxarm.org/platforms/armv6/raspberry-pi}{here},


\section{command lines}
\label{\detokenize{integration:command-lines}}
\sphinxAtStartPar
like this: \sphinxcode{\sphinxupquote{192.168.0.x}}, \sphinxcode{\sphinxupquote{10.0.0.14x}} or

\sphinxAtStartPar
Enough of networking for now. We’ll set a proper network configuration later in this guide, but first some \sphinxstyleemphasis{musthaves}.


\subsection{text block}
\label{\detokenize{integration:text-block}}
\begin{sphinxVerbatim}[commandchars=\\\{\}]
\PYG{n}{passwd}  \PYG{c+c1}{\PYGZsh{} change root password to something important}
\PYG{n}{rm} \PYG{o}{\PYGZhy{}}\PYG{n}{rf} \PYG{o}{/}\PYG{n}{etc}\PYG{o}{/}\PYG{n}{localtime}  \PYG{c+c1}{\PYGZsh{} dont care about this}
\PYG{n}{ln} \PYG{o}{\PYGZhy{}}\PYG{n}{s} \PYG{o}{/}\PYG{n}{usr}\PYG{o}{/}\PYG{n}{share}\PYG{o}{/}\PYG{n}{zoneinfo}\PYG{o}{/}\PYG{n}{Europe}\PYG{o}{/}\PYG{n}{Prague} \PYG{o}{/}\PYG{n}{etc}\PYG{o}{/}\PYG{n}{localtime}  \PYG{c+c1}{\PYGZsh{} set appropriate timezone}
\PYG{n}{echo} \PYG{l+s+s2}{\PYGZdq{}}\PYG{l+s+s2}{my\PYGZus{}raspberry}\PYG{l+s+s2}{\PYGZdq{}} \PYG{o}{\PYGZgt{}}  \PYG{o}{/}\PYG{n}{etc}\PYG{o}{/}\PYG{n}{hostname}  \PYG{c+c1}{\PYGZsh{} set name of your RPi}

\PYG{n}{useradd} \PYG{o}{\PYGZhy{}}\PYG{n}{m} \PYG{o}{\PYGZhy{}}\PYG{n}{aG} \PYG{n}{wheel} \PYG{o}{\PYGZhy{}}\PYG{n}{s} \PYG{o}{/}\PYG{n}{usr}\PYG{o}{/}\PYG{n+nb}{bin}\PYG{o}{/}\PYG{n}{bash} \PYG{n}{common\PYGZus{}user} \PYG{c+c1}{\PYGZsh{}}
\PYG{n}{groupadd} \PYG{n}{webdata}  \PYG{c+c1}{\PYGZsh{} for sharing}
\PYG{n}{useradd} \PYG{o}{\PYGZhy{}}\PYG{n}{M} \PYG{o}{\PYGZhy{}}\PYG{n}{aG} \PYG{n}{webdata} \PYG{o}{\PYGZhy{}}\PYG{n}{s} \PYG{o}{/}\PYG{n}{usr}\PYG{o}{/}\PYG{n+nb}{bin}\PYG{o}{/}\PYG{n}{false} \PYG{n}{nginx}
\PYG{n}{usermod} \PYG{o}{\PYGZhy{}}\PYG{n}{aG} \PYG{n}{webdata} \PYG{n}{common\PYGZus{}user}

\PYG{n}{visudo}  \PYG{c+c1}{\PYGZsh{} uncomment this line:  \PYGZpc{}wheel ALL=(ALL) ALL}

\PYG{n}{pacman} \PYG{o}{\PYGZhy{}}\PYG{n}{Syu}
\end{sphinxVerbatim}

\sphinxAtStartPar
\sphinxstylestrong{bold text}
\begin{itemize}
\item {} 
\sphinxAtStartPar
bullet

\item {} 
\sphinxAtStartPar
point

\end{itemize}


\chapter{Upgrading}
\label{\detokenize{upgrading:upgrading}}\label{\detokenize{upgrading::doc}}
\sphinxAtStartPar
The term ‘Upgrade’ is a term reserved for an alteration made to the software build itself, in which case an increase to the second but last digit is made to reflect this e.g. \sphinxcode{\sphinxupquote{0.X.0}}.  An Upgrade will more than likely require the user to re\sphinxhyphen{}download and install the software. The exception to this rule is in the case of major updates which can change the user experience so dramatically we mark the occaSiôn by changing the second but one digit of the version (as we do in the case of upgrades e.g. \sphinxcode{\sphinxupquote{0.X.0}})


\chapter{Troubleshooting}
\label{\detokenize{troubleshooting:troubleshooting}}\label{\detokenize{troubleshooting::doc}}

\section{hyperlink}
\label{\detokenize{troubleshooting:hyperlink}}
\sphinxAtStartPar
Hyperlink \sphinxhref{https://ArchLinuxarm.org/platforms/armv6/raspberry-pi}{here},


\section{command lines}
\label{\detokenize{troubleshooting:command-lines}}
\sphinxAtStartPar
like this: \sphinxcode{\sphinxupquote{192.168.0.x}}, \sphinxcode{\sphinxupquote{10.0.0.14x}} or

\sphinxAtStartPar
Enough of networking for now. We’ll set a proper network configuration later in this guide, but first some \sphinxstyleemphasis{musthaves}.


\subsection{text block}
\label{\detokenize{troubleshooting:text-block}}
\begin{sphinxVerbatim}[commandchars=\\\{\}]
\PYG{n}{passwd}  \PYG{c+c1}{\PYGZsh{} change root password to something important}
\PYG{n}{rm} \PYG{o}{\PYGZhy{}}\PYG{n}{rf} \PYG{o}{/}\PYG{n}{etc}\PYG{o}{/}\PYG{n}{localtime}  \PYG{c+c1}{\PYGZsh{} dont care about this}
\PYG{n}{ln} \PYG{o}{\PYGZhy{}}\PYG{n}{s} \PYG{o}{/}\PYG{n}{usr}\PYG{o}{/}\PYG{n}{share}\PYG{o}{/}\PYG{n}{zoneinfo}\PYG{o}{/}\PYG{n}{Europe}\PYG{o}{/}\PYG{n}{Prague} \PYG{o}{/}\PYG{n}{etc}\PYG{o}{/}\PYG{n}{localtime}  \PYG{c+c1}{\PYGZsh{} set appropriate timezone}
\PYG{n}{echo} \PYG{l+s+s2}{\PYGZdq{}}\PYG{l+s+s2}{my\PYGZus{}raspberry}\PYG{l+s+s2}{\PYGZdq{}} \PYG{o}{\PYGZgt{}}  \PYG{o}{/}\PYG{n}{etc}\PYG{o}{/}\PYG{n}{hostname}  \PYG{c+c1}{\PYGZsh{} set name of your RPi}

\PYG{n}{useradd} \PYG{o}{\PYGZhy{}}\PYG{n}{m} \PYG{o}{\PYGZhy{}}\PYG{n}{aG} \PYG{n}{wheel} \PYG{o}{\PYGZhy{}}\PYG{n}{s} \PYG{o}{/}\PYG{n}{usr}\PYG{o}{/}\PYG{n+nb}{bin}\PYG{o}{/}\PYG{n}{bash} \PYG{n}{common\PYGZus{}user} \PYG{c+c1}{\PYGZsh{}}
\PYG{n}{groupadd} \PYG{n}{webdata}  \PYG{c+c1}{\PYGZsh{} for sharing}
\PYG{n}{useradd} \PYG{o}{\PYGZhy{}}\PYG{n}{M} \PYG{o}{\PYGZhy{}}\PYG{n}{aG} \PYG{n}{webdata} \PYG{o}{\PYGZhy{}}\PYG{n}{s} \PYG{o}{/}\PYG{n}{usr}\PYG{o}{/}\PYG{n+nb}{bin}\PYG{o}{/}\PYG{n}{false} \PYG{n}{nginx}
\PYG{n}{usermod} \PYG{o}{\PYGZhy{}}\PYG{n}{aG} \PYG{n}{webdata} \PYG{n}{common\PYGZus{}user}

\PYG{n}{visudo}  \PYG{c+c1}{\PYGZsh{} uncomment this line:  \PYGZpc{}wheel ALL=(ALL) ALL}

\PYG{n}{pacman} \PYG{o}{\PYGZhy{}}\PYG{n}{Syu}
\end{sphinxVerbatim}

\sphinxAtStartPar
\sphinxstylestrong{bold text}
\begin{itemize}
\item {} 
\sphinxAtStartPar
bullet

\item {} 
\sphinxAtStartPar
point

\end{itemize}


\chapter{PDF’s \& Video’s}
\label{\detokenize{docsandvideos:pdf-s-video-s}}\label{\detokenize{docsandvideos::doc}}

\section{hyperlink}
\label{\detokenize{docsandvideos:hyperlink}}
\sphinxAtStartPar
Hyperlink \sphinxhref{https://ArchLinuxarm.org/platforms/armv6/raspberry-pi}{here},


\section{command lines}
\label{\detokenize{docsandvideos:command-lines}}
\sphinxAtStartPar
like this: \sphinxcode{\sphinxupquote{192.168.0.x}}, \sphinxcode{\sphinxupquote{10.0.0.14x}} or

\sphinxAtStartPar
Enough of networking for now. We’ll set a proper network configuration later in this guide, but first some \sphinxstyleemphasis{musthaves}.


\subsection{text block}
\label{\detokenize{docsandvideos:text-block}}
\begin{sphinxVerbatim}[commandchars=\\\{\}]
\PYG{n}{passwd}  \PYG{c+c1}{\PYGZsh{} change root password to something important}
\PYG{n}{rm} \PYG{o}{\PYGZhy{}}\PYG{n}{rf} \PYG{o}{/}\PYG{n}{etc}\PYG{o}{/}\PYG{n}{localtime}  \PYG{c+c1}{\PYGZsh{} dont care about this}
\PYG{n}{ln} \PYG{o}{\PYGZhy{}}\PYG{n}{s} \PYG{o}{/}\PYG{n}{usr}\PYG{o}{/}\PYG{n}{share}\PYG{o}{/}\PYG{n}{zoneinfo}\PYG{o}{/}\PYG{n}{Europe}\PYG{o}{/}\PYG{n}{Prague} \PYG{o}{/}\PYG{n}{etc}\PYG{o}{/}\PYG{n}{localtime}  \PYG{c+c1}{\PYGZsh{} set appropriate timezone}
\PYG{n}{echo} \PYG{l+s+s2}{\PYGZdq{}}\PYG{l+s+s2}{my\PYGZus{}raspberry}\PYG{l+s+s2}{\PYGZdq{}} \PYG{o}{\PYGZgt{}}  \PYG{o}{/}\PYG{n}{etc}\PYG{o}{/}\PYG{n}{hostname}  \PYG{c+c1}{\PYGZsh{} set name of your RPi}

\PYG{n}{useradd} \PYG{o}{\PYGZhy{}}\PYG{n}{m} \PYG{o}{\PYGZhy{}}\PYG{n}{aG} \PYG{n}{wheel} \PYG{o}{\PYGZhy{}}\PYG{n}{s} \PYG{o}{/}\PYG{n}{usr}\PYG{o}{/}\PYG{n+nb}{bin}\PYG{o}{/}\PYG{n}{bash} \PYG{n}{common\PYGZus{}user} \PYG{c+c1}{\PYGZsh{}}
\PYG{n}{groupadd} \PYG{n}{webdata}  \PYG{c+c1}{\PYGZsh{} for sharing}
\PYG{n}{useradd} \PYG{o}{\PYGZhy{}}\PYG{n}{M} \PYG{o}{\PYGZhy{}}\PYG{n}{aG} \PYG{n}{webdata} \PYG{o}{\PYGZhy{}}\PYG{n}{s} \PYG{o}{/}\PYG{n}{usr}\PYG{o}{/}\PYG{n+nb}{bin}\PYG{o}{/}\PYG{n}{false} \PYG{n}{nginx}
\PYG{n}{usermod} \PYG{o}{\PYGZhy{}}\PYG{n}{aG} \PYG{n}{webdata} \PYG{n}{common\PYGZus{}user}

\PYG{n}{visudo}  \PYG{c+c1}{\PYGZsh{} uncomment this line:  \PYGZpc{}wheel ALL=(ALL) ALL}

\PYG{n}{pacman} \PYG{o}{\PYGZhy{}}\PYG{n}{Syu}
\end{sphinxVerbatim}

\sphinxAtStartPar
\sphinxstylestrong{bold text}
\begin{itemize}
\item {} 
\sphinxAtStartPar
bullet

\item {} 
\sphinxAtStartPar
point

\end{itemize}


\chapter{FAQ’s and Other Resources}
\label{\detokenize{faq:faq-s-and-other-resources}}\label{\detokenize{faq::doc}}

\section{hyperlink}
\label{\detokenize{faq:hyperlink}}
\sphinxAtStartPar
Hyperlink \sphinxhref{https://ArchLinuxarm.org/platforms/armv6/raspberry-pi}{here},


\section{command lines}
\label{\detokenize{faq:command-lines}}
\sphinxAtStartPar
like this: \sphinxcode{\sphinxupquote{192.168.0.x}}, \sphinxcode{\sphinxupquote{10.0.0.14x}} or

\sphinxAtStartPar
Enough of networking for now. We’ll set a proper network configuration later in this guide, but first some \sphinxstyleemphasis{musthaves}.


\subsection{text block}
\label{\detokenize{faq:text-block}}
\begin{sphinxVerbatim}[commandchars=\\\{\}]
\PYG{n}{passwd}  \PYG{c+c1}{\PYGZsh{} change root password to something important}
\PYG{n}{rm} \PYG{o}{\PYGZhy{}}\PYG{n}{rf} \PYG{o}{/}\PYG{n}{etc}\PYG{o}{/}\PYG{n}{localtime}  \PYG{c+c1}{\PYGZsh{} dont care about this}
\PYG{n}{ln} \PYG{o}{\PYGZhy{}}\PYG{n}{s} \PYG{o}{/}\PYG{n}{usr}\PYG{o}{/}\PYG{n}{share}\PYG{o}{/}\PYG{n}{zoneinfo}\PYG{o}{/}\PYG{n}{Europe}\PYG{o}{/}\PYG{n}{Prague} \PYG{o}{/}\PYG{n}{etc}\PYG{o}{/}\PYG{n}{localtime}  \PYG{c+c1}{\PYGZsh{} set appropriate timezone}
\PYG{n}{echo} \PYG{l+s+s2}{\PYGZdq{}}\PYG{l+s+s2}{my\PYGZus{}raspberry}\PYG{l+s+s2}{\PYGZdq{}} \PYG{o}{\PYGZgt{}}  \PYG{o}{/}\PYG{n}{etc}\PYG{o}{/}\PYG{n}{hostname}  \PYG{c+c1}{\PYGZsh{} set name of your RPi}

\PYG{n}{useradd} \PYG{o}{\PYGZhy{}}\PYG{n}{m} \PYG{o}{\PYGZhy{}}\PYG{n}{aG} \PYG{n}{wheel} \PYG{o}{\PYGZhy{}}\PYG{n}{s} \PYG{o}{/}\PYG{n}{usr}\PYG{o}{/}\PYG{n+nb}{bin}\PYG{o}{/}\PYG{n}{bash} \PYG{n}{common\PYGZus{}user} \PYG{c+c1}{\PYGZsh{}}
\PYG{n}{groupadd} \PYG{n}{webdata}  \PYG{c+c1}{\PYGZsh{} for sharing}
\PYG{n}{useradd} \PYG{o}{\PYGZhy{}}\PYG{n}{M} \PYG{o}{\PYGZhy{}}\PYG{n}{aG} \PYG{n}{webdata} \PYG{o}{\PYGZhy{}}\PYG{n}{s} \PYG{o}{/}\PYG{n}{usr}\PYG{o}{/}\PYG{n+nb}{bin}\PYG{o}{/}\PYG{n}{false} \PYG{n}{nginx}
\PYG{n}{usermod} \PYG{o}{\PYGZhy{}}\PYG{n}{aG} \PYG{n}{webdata} \PYG{n}{common\PYGZus{}user}

\PYG{n}{visudo}  \PYG{c+c1}{\PYGZsh{} uncomment this line:  \PYGZpc{}wheel ALL=(ALL) ALL}

\PYG{n}{pacman} \PYG{o}{\PYGZhy{}}\PYG{n}{Syu}
\end{sphinxVerbatim}

\sphinxAtStartPar
\sphinxstylestrong{bold text}
\begin{itemize}
\item {} 
\sphinxAtStartPar
bullet

\item {} 
\sphinxAtStartPar
point

\end{itemize}


\chapter{Developers}
\label{\detokenize{developers:developers}}\label{\detokenize{developers::doc}}

\section{hyperlink}
\label{\detokenize{developers:hyperlink}}
\sphinxAtStartPar
Hyperlink \sphinxhref{https://ArchLinuxarm.org/platforms/armv6/raspberry-pi}{here},


\section{command lines}
\label{\detokenize{developers:command-lines}}
\sphinxAtStartPar
like this: \sphinxcode{\sphinxupquote{192.168.0.x}}, \sphinxcode{\sphinxupquote{10.0.0.14x}} or

\sphinxAtStartPar
Enough of networking for now. We’ll set a proper network configuration later in this guide, but first some \sphinxstyleemphasis{musthaves}.


\subsection{text block}
\label{\detokenize{developers:text-block}}
\begin{sphinxVerbatim}[commandchars=\\\{\}]
\PYG{n}{passwd}  \PYG{c+c1}{\PYGZsh{} change root password to something important}
\PYG{n}{rm} \PYG{o}{\PYGZhy{}}\PYG{n}{rf} \PYG{o}{/}\PYG{n}{etc}\PYG{o}{/}\PYG{n}{localtime}  \PYG{c+c1}{\PYGZsh{} dont care about this}
\PYG{n}{ln} \PYG{o}{\PYGZhy{}}\PYG{n}{s} \PYG{o}{/}\PYG{n}{usr}\PYG{o}{/}\PYG{n}{share}\PYG{o}{/}\PYG{n}{zoneinfo}\PYG{o}{/}\PYG{n}{Europe}\PYG{o}{/}\PYG{n}{Prague} \PYG{o}{/}\PYG{n}{etc}\PYG{o}{/}\PYG{n}{localtime}  \PYG{c+c1}{\PYGZsh{} set appropriate timezone}
\PYG{n}{echo} \PYG{l+s+s2}{\PYGZdq{}}\PYG{l+s+s2}{my\PYGZus{}raspberry}\PYG{l+s+s2}{\PYGZdq{}} \PYG{o}{\PYGZgt{}}  \PYG{o}{/}\PYG{n}{etc}\PYG{o}{/}\PYG{n}{hostname}  \PYG{c+c1}{\PYGZsh{} set name of your RPi}

\PYG{n}{useradd} \PYG{o}{\PYGZhy{}}\PYG{n}{m} \PYG{o}{\PYGZhy{}}\PYG{n}{aG} \PYG{n}{wheel} \PYG{o}{\PYGZhy{}}\PYG{n}{s} \PYG{o}{/}\PYG{n}{usr}\PYG{o}{/}\PYG{n+nb}{bin}\PYG{o}{/}\PYG{n}{bash} \PYG{n}{common\PYGZus{}user} \PYG{c+c1}{\PYGZsh{}}
\PYG{n}{groupadd} \PYG{n}{webdata}  \PYG{c+c1}{\PYGZsh{} for sharing}
\PYG{n}{useradd} \PYG{o}{\PYGZhy{}}\PYG{n}{M} \PYG{o}{\PYGZhy{}}\PYG{n}{aG} \PYG{n}{webdata} \PYG{o}{\PYGZhy{}}\PYG{n}{s} \PYG{o}{/}\PYG{n}{usr}\PYG{o}{/}\PYG{n+nb}{bin}\PYG{o}{/}\PYG{n}{false} \PYG{n}{nginx}
\PYG{n}{usermod} \PYG{o}{\PYGZhy{}}\PYG{n}{aG} \PYG{n}{webdata} \PYG{n}{common\PYGZus{}user}

\PYG{n}{visudo}  \PYG{c+c1}{\PYGZsh{} uncomment this line:  \PYGZpc{}wheel ALL=(ALL) ALL}

\PYG{n}{pacman} \PYG{o}{\PYGZhy{}}\PYG{n}{Syu}
\end{sphinxVerbatim}

\sphinxAtStartPar
\sphinxstylestrong{bold text}
\begin{itemize}
\item {} 
\sphinxAtStartPar
bullet

\item {} 
\sphinxAtStartPar
point

\end{itemize}


\chapter{Housekeeping}
\label{\detokenize{networkmaint:housekeeping}}\label{\detokenize{networkmaint::doc}}

\section{hyperlink}
\label{\detokenize{networkmaint:hyperlink}}
\sphinxAtStartPar
Hyperlink \sphinxhref{https://ArchLinuxarm.org/platforms/armv6/raspberry-pi}{here},


\section{command lines}
\label{\detokenize{networkmaint:command-lines}}
\sphinxAtStartPar
like this: \sphinxcode{\sphinxupquote{192.168.0.x}}, \sphinxcode{\sphinxupquote{10.0.0.14x}} or

\sphinxAtStartPar
Enough of networking for now. We’ll set a proper network configuration later in this guide, but first some \sphinxstyleemphasis{musthaves}.


\subsection{text block}
\label{\detokenize{networkmaint:text-block}}
\begin{sphinxVerbatim}[commandchars=\\\{\}]
\PYG{n}{passwd}  \PYG{c+c1}{\PYGZsh{} change root password to something important}
\PYG{n}{rm} \PYG{o}{\PYGZhy{}}\PYG{n}{rf} \PYG{o}{/}\PYG{n}{etc}\PYG{o}{/}\PYG{n}{localtime}  \PYG{c+c1}{\PYGZsh{} dont care about this}
\PYG{n}{ln} \PYG{o}{\PYGZhy{}}\PYG{n}{s} \PYG{o}{/}\PYG{n}{usr}\PYG{o}{/}\PYG{n}{share}\PYG{o}{/}\PYG{n}{zoneinfo}\PYG{o}{/}\PYG{n}{Europe}\PYG{o}{/}\PYG{n}{Prague} \PYG{o}{/}\PYG{n}{etc}\PYG{o}{/}\PYG{n}{localtime}  \PYG{c+c1}{\PYGZsh{} set appropriate timezone}
\PYG{n}{echo} \PYG{l+s+s2}{\PYGZdq{}}\PYG{l+s+s2}{my\PYGZus{}raspberry}\PYG{l+s+s2}{\PYGZdq{}} \PYG{o}{\PYGZgt{}}  \PYG{o}{/}\PYG{n}{etc}\PYG{o}{/}\PYG{n}{hostname}  \PYG{c+c1}{\PYGZsh{} set name of your RPi}

\PYG{n}{useradd} \PYG{o}{\PYGZhy{}}\PYG{n}{m} \PYG{o}{\PYGZhy{}}\PYG{n}{aG} \PYG{n}{wheel} \PYG{o}{\PYGZhy{}}\PYG{n}{s} \PYG{o}{/}\PYG{n}{usr}\PYG{o}{/}\PYG{n+nb}{bin}\PYG{o}{/}\PYG{n}{bash} \PYG{n}{common\PYGZus{}user} \PYG{c+c1}{\PYGZsh{}}
\PYG{n}{groupadd} \PYG{n}{webdata}  \PYG{c+c1}{\PYGZsh{} for sharing}
\PYG{n}{useradd} \PYG{o}{\PYGZhy{}}\PYG{n}{M} \PYG{o}{\PYGZhy{}}\PYG{n}{aG} \PYG{n}{webdata} \PYG{o}{\PYGZhy{}}\PYG{n}{s} \PYG{o}{/}\PYG{n}{usr}\PYG{o}{/}\PYG{n+nb}{bin}\PYG{o}{/}\PYG{n}{false} \PYG{n}{nginx}
\PYG{n}{usermod} \PYG{o}{\PYGZhy{}}\PYG{n}{aG} \PYG{n}{webdata} \PYG{n}{common\PYGZus{}user}

\PYG{n}{visudo}  \PYG{c+c1}{\PYGZsh{} uncomment this line:  \PYGZpc{}wheel ALL=(ALL) ALL}

\PYG{n}{pacman} \PYG{o}{\PYGZhy{}}\PYG{n}{Syu}
\end{sphinxVerbatim}

\sphinxAtStartPar
\sphinxstylestrong{bold text}
\begin{itemize}
\item {} 
\sphinxAtStartPar
bullet

\item {} 
\sphinxAtStartPar
point

\end{itemize}


\chapter{\sphinxstylestrong{Document Author(s):}}
\label{\detokenize{index:document-author-s}}

\section{DATRO Consortium}
\label{\detokenize{index:datro-consortium}}


\renewcommand{\indexname}{Index}
\printindex
\end{document}